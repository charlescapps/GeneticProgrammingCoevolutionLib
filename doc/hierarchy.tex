\documentclass{article}

\usepackage{fullpage, graphicx, amssymb, amsmath, amsfonts, wrapfig, subfigure, listings, color, hyperref, multicol}
\usepackage[usenames,dvipsnames]{xcolor}
\usepackage{tikz}
\usetikzlibrary{positioning,shapes,shadows,arrows}
\newcommand{\nl}{\vspace{\baselineskip}}
\newcommand{\tab}{\hspace*{2em}}
\newcommand{\bo}[1]{\textbf{#1}}
\newcommand{\tm}[1]{\textrm{#1}}
\newcommand{\imp}{\longrightarrow}
\newcommand{\f}{\forall}
\newcommand{\e}{\exists}
\newcommand{\AND}{\wedge}
\newcommand{\OR}{\vee}

%Some settings to make lstlistings pretty
\lstset{ 
language=C,                % choose the language of the code
basicstyle=\footnotesize,       % the size of the fonts that are used for the code
numbers=left,                   % where to put the line-numbers
numberstyle=\footnotesize,      % the size of the fonts that are used for the line-numbers
stepnumber=1,                   % the step between two line-numbers. If it is 1 each line will be numbered
numbersep=5pt,                  % how far the line-numbers are from the code
backgroundcolor=\color{white},  % choose the background color. You must add \usepackage{color}
showspaces=false,               % show spaces adding particular underscores
showstringspaces=false,         % underline spaces within strings
showtabs=false,                 % show tabs within strings adding particular underscores
frame=single,   		% adds a frame around the code
tabsize=2,  		% sets default tabsize to 2 spaces
captionpos=b,   		% sets the caption-position to bottom
breaklines=true,    	% sets automatic line breaking
breakatwhitespace=false,    % sets if automatic breaks should only happen at whitespace
%escapeinside={\%}{)}          % if you want to add a comment within your code
}

\begin{document}
\tikzstyle{abstract}=[rectangle, draw=black, rounded corners, fill=blue!40, drop shadow,
        text centered, anchor=north, text=white, text width=3cm]
\tikzstyle{comment}=[rectangle, draw=black, rounded corners, fill=green, drop shadow,
        text centered, anchor=north, text=white, text width=3cm]
\tikzstyle{myarrow}=[->, >=open triangle 90, thick]
\tikzstyle{line}=[-, thick]

\begin{center}
\LARGE{\bo{Genetic Programming Applied to Minichess}} \\
\Large{Charles L. Capps\\
Portland State University} \\ 
\today \\

\end{center}
\section{Class hierarchy}
\begin{center}
\begin{tikzpicture}[node distance=1.5cm]
%ROOT
    \node (gpnode) [abstract, rectangle split, rectangle split parts=3, text width=8cm]
        {

            \textbf{abstract class GPNode} 
		    \nodepart{second}
		    private \verb=List<GPNode> subtrees=
		    \nodepart{third} 
		    \begin{flushleft}
		    \color{white}
		    abstract int \verb=interpretForValue(GameState)=\\
		
		    abstract int \verb=numSubtrees()= //how many to attach\\
		    abstract String \verb=label()= //for creating dot file\\
		    \nl
		    \verb=List<GPNode> subtrees()= \\
		    void \verb=attachSubtrees(List<GPNode>)= \\
		    String \verb=toDot()= //use graphviz to visualize
	    \end{flushleft}
        };
%Depth 2
    \node (AuxNode01) [text width=0.1cm, below=of gpnode] {};
    \node (gpfunctionnode) [abstract, rectangle split, rectangle split parts=2, left=of AuxNode01, text width=6cm]
        {
            \textbf{abstract class GPFunction}
            \nodepart{second}
            abstract \verb=int numSubtrees= //number of args to function
        };

    \node (gptermnode) [abstract, rectangle split, rectangle split parts=2, right=of AuxNode01, text width=7cm]
        {
            \textbf{abstract class GPTerminal}
            \nodepart{second}
            int \verb=numSubtrees()= //returns 0 \\
            \verb=List<GPNode> subtrees()= //returns null
        };
%Depth 3      

\node (AuxNode02) [text width=1cm, below=2.5 cm of AuxNode01] {};

    \node (erc) [comment, rectangle split, rectangle split parts=2, right=1.0cm of AuxNode02, text width=3cm]
        {
            \textbf{ERCNode} \nodepart{second} \bo{random const}
        };
    \node (pawns) [comment, rectangle split, rectangle split parts=2, right=0.5cm of erc, text width=3.5cm]
        {
            \textbf{NumPawnsSensor}
            \nodepart{second}
            \bo{//counts pawns}
        };
        
    \node (plus) [comment, rectangle split, rectangle split parts=2, left=0.5cm of AuxNode02, text width=3cm]
        {
            \textbf{PlusFunc} 
            \nodepart{second} \bo{binary function}
        };
    \node (times) [comment, rectangle split, rectangle split parts=2, left=0.5cm of plus, text width=3.0cm]
        {
            \textbf{TimesFunc}
            \nodepart{second} \bo{binary function}
        };

%Edges        
    \draw[myarrow] (gpfunctionnode.north) -- ++(0,0.4) -| (gpnode.south);
    \draw[line] (gpfunctionnode.north) -- ++(0,0.4) -| (gptermnode.north);

    \draw[myarrow] (erc.north) -- ++(0,0.4) -| (gptermnode.south);
    \draw[line] (erc.north) -- ++(0,0.4) -| (pawns.north);

    \draw[myarrow] (times.north) -- ++(0,0.4) -| (gpfunctionnode.south);
    \draw[line] (times.north) -- ++(0,0.4) -| (plus.north);
        
%Singleton classes
    \node (gamestate) [abstract, rectangle split, rectangle split parts=2, text width=8cm, 
    	below = 1.5cm of AuxNode02]
        {
        \textbf{interface GameState}
        \nodepart{second}
        \bo{Dummy interface. Concrete node class must verify the runtime type is correct and perform downcast.}
        };
        
\node (gptree) [abstract, rectangle split, rectangle split parts=3, text width=8cm, 
    	below = 0.5cm of gamestate]
        {
        \textbf{class GPTree - \\ 
        generate random GPTree \\ 
        This works independent of domain used.}
        \nodepart{second}
        \verb=private GPNode root=\\
        \verb=public static enum METHOD {GROW, FULL}=
        \nodepart{third}
        \verb=public GPTree(int maxDepth, METHOD m,= \\
        \verb=List<Class> funcs, List<Class> terms)=
        };
\end{tikzpicture}
\end{center}

 
\end{document} 
